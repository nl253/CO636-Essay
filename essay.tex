%% options: [no]titlepage, twocolumn, landscape, draft
\documentclass[a4paper, 12pt, titlepage]{article}
\usepackage{graphicx}
\usepackage{titlesec}

\titlespacing*{\section}
{0pt}{1.0ex plus 1ex minus .2ex}{1.0ex plus .2ex}
\titlespacing*{\subsection}
{0pt}{1.0ex plus 1ex minus .2ex}{1.0ex plus .2ex}
\usepackage[utf8]{inputenc}
\usepackage[english]{babel}
\usepackage[margin=0.35in]{geometry}

\usepackage{minted}
\date{} % overwrite default format
\author{Norbert Logiewa\\nl253}
\title{Applications of Evolutionary Algorithms in Scheduling}

\begin{document}

\maketitle

Optimal scheduling has been classed as NP-hard \cite[p.~26]{aljarrah2017}
\cite[p.~52]{diveev2017} \cite[p.~92]{mesghouni2004}.  To find a
satisfactory solution we must resort to non-standard approaches such
as Evolutionary Algorithms (EA). EAs are often employed when dealing
with multi-dimensional combinatorial optimization problems with a large
search space, such as this one, where sampling of every candidate is not
possible \cite[p.~29]{floreano2008} \cite[p.~53]{diveev2017}.  EAs are
a good choice in such scenarios because they allow to find good-enough
solution to problems that don't necessarily need an \emph{optimal}
solution\cite[p.~44]{heaton2014} \cite[p.~53]{diveev2017}.  They also
present an opportunity to exploit the computing power of modern multi-core
CPUs by running subroutines in parallel. \cite[p.~28]{aljarrah2017}
This essay will explore how GAs can be applied to solve scheduling problems.

\section*{The Problem}

\emph{Scheduling} in this essay will refer to ordering of tasks that aims to maximise the
productivity whilst adhering to the constraints that an organisation
may impose.  The ordering may schedule some activities to take place in
parallel if business rules and the amount of resources allow for that.
Efficient scheduling of tasks if of interests to anyone who wishes to
increase their productivity \emph{without} investing additional resources.

Scheduling for an organisation is often associated with adhering to a
set of constraints that must be met called \emph{hard constraints} in
the literature. These would be things as: lecturers can only give one lecture at a time.
There are also \emph{soft constraints} which benefit candidates that
adhere to them but are not compulsory. E.g. the timetable includes breaks
in between activities to allow for rest. \cite[p.~26]{aljarrah2017}
\cite{timilsina2015} \cite[p.~2505]{qaurooni2013} \cite[p.~56]{diveev2017}

\section*{Solving the Problem}

Applying an EA entails the use of 3 main components: a selection
operator \footnote{to remove the unfit candidates from the population}, genetic operator \footnote{to generate new candidate solutions}
and a fitness function \footnote{to calculate how fit the candidates are}
\cite{mesghouni2004, yu2002, diveev2017, aljarrah2017, timilsina2015}.

\subsection*{Encoding Candidates}

% TODO issues with the current approach (if applicable) or why the current approach is good
% TODO conclude suggesting best encoding 

In GAs candidates are most commonly encoded using bit-strings
\cite[p.~127]{norvig2010} and this encoding has received the most
attention in the literature \cite[p.~103]{eberhart2007}.
However, it is not certain that this is the most natural, the most
convenient or the most efficient encoding for a scheduling problem.
The standard bit-string encoding has been criticised by Mesghouni et al.
who say ``[...] this approach cannot usually be used for real word engineering problems such as combinatorial ones [...]''\cite[p.~93]{mesghouni2004}. Instead they suggest the use of numbers to represent every gene
so that each candidate is a permutation of numbers.
The use of numbers instead of bit-strings has also been endorsed by May et al.
and Solanki who also used numbers to encode their candidate solutions. \cite[p.~7076]{may2015} \cite[p.~3868]{solanki2015}
Researchers who wish to abandon the traditional approach might need to
invent new operators or adapt existing ones to fit their encoding scheme
which may be seen as a disadvantage.

\subsection*{Selection Operator}

Selection is the core of every genetic algorithm. A good selection scheme
ensures the algorithm does not converge prematurely on a sub-optimal
solution because the selection pressure was too high (and the diversity was
lost prematurely \cite[p.~23]{floreano2008}) and at the same
time does not let the procedure run for too long. \cite{yu2002}

The GA literature discusses a number of selection operators including:
proportionate \footnote{also known as \emph{roulette wheel} selection}, rank-based and
tournament among others \cite{floreano2008, eberhart2007, heaton2014}.

A popular choice seems to be proportionate selection with
many researchers discussing and implementing GAs with this
scheme. \cite{timilsina2015, yu2002, aljarrah2017, mesghouni2004}
Al-Jarrah et al. who set out to create a timetable for a university, reported
excellent results when using proportionate selection scheme
with instructors rating the output timetable as \(9.56/10\).
This suggests that proportionate selection is a good choice of selection operator.

However another, potentially better choice could be rank-based
selection (RBS).  RBS avoids issues that proportionate selection faces:
when all candidates have a similar fitness or one candidate has a
large high fitness score, the selection turns into a random search.
\cite[p.~23]{floreano2008}. Unfortunately, there does not seem to be
much research that wishes to test this hypothesis and see if it would
yield better results.

Furthermore, there are reasons to believe that tournament selection would
also be beneficial.  Since each tournament is run on a number of distinct
individuals, each one of those tournaments can be run independently
meaning we can execute them in parallel. This might address the issue
of poor performance. 

\subsection*{Genetic Operators}

In terms of genetic operators research into GAs and scheduling seems
to rely mainly on crossover and mutation \cite{yu2002, aljarrah2017,
mesghouni2004, hosseinabadi2018, solanki2015, timilsina2015, may2015}.
The aim of selection operators is to reduce the size of the population. To
replenish it, winners are used as input to genetic operators to create
new individuals.  Crossover is commonly used because it is a standard and
well-understood operator. Additionally, mutation with a small probability
tends to be introduced to further increase variation in the population.
These two tend to be combined because cross-over \emph{recombines}
existing genes whereas mutation introduces \emph{novelty}.
\cite[p.~27]{heaton2014}.  A good starting values for the parameters would
be as follows: \(p(crossover) = 0.8\) \cite[p.~117]{eberhart2007} and
\(p(mutation) = 0.2\) \cite[p.~25]{heaton2014} although some researchers
suggest much lower rates e.g. \(0.005\) \cite[p.~117]{eberhart2007}.

The use of the standard operators has been criticised in timetabling
problems.  By looking at a single candidate \footnote{mutation is typically a unary operator} or a
pair \footnote{cross-over is typically a binary operator} we are risking crating an invalid candidate that breaks
hard constraints. We might, for instance, schedule a student to be in two
places at the same time \emph{unless} when generating candidates we consider the candidates that have been created
so far. \cite[p.~2506]{qaurooni2013}.  This may save us from the danger
of generating pointless candidates. Qaurooni and Akbarzadeh-T argue this
saves us space and without group-based operators ``the search space of a
grouping problem grows much larger than it has to be [...]''.
However, the introduction of group-based operators may come at a cost. Firstly,
the complexity of the algorithm increases.  Many are drawn to
GAs \emph{because} they are simple.  Secondly, in this scheme we
would need to iterate through the set of crated candidates every time a
genetic operator is applied and carry out (potentially expensive) checks.
Further research needs to be done to determine if alternative operators
should be preferred. It might be that group-based operators save time but
also waste time on performing all the checks for meeting hard-constraints.

\subsection*{Fitness Function}

The fitness function (FF) is problem-specific. Therefore, it is difficult
to give any specific guidelines with regard to how it should be
designed. Along with the parameters such as initial population size the
FF is one of the main concerns programmers face when using
genetic algorithms. In essence, FF measures to what
extent a candidate solution obeys the soft and hard constraints.

% There is likely many good-enough solutions to the problem i.e. many
% timetables could satisfy the constraints listed above. In other words,
% the fitness landscape has many peaks.

\subsection*{Termination}

% TODO what tends to be the ending criterion
% TODO issues with the current approach (if applicable) or why the current approach is good
% TODO conclude suggesting best invariant (timeout)

GA is run repeated until either enough time has passed or other condition
has been met. \cite[p.~108]{eberhart2007}.
Many researchers chose to wait for a fixed number of iterations before
terminating execution of the algorithm. \cite[p.~3868]{solanki2015} \cite[p.~7078]{may2015} \cite{hosseinabadi2018}

It's also possible to wait for a candidate with a satisfactory fitness score as Solanki suggests. \cite[p.~3868]{solanki2015}. In some cases this might save us time if the algorithm happens to be initialised in the global optimum, for instance. However, this is very unlikely and the iteration-based timeout would need to remain otherwise there is a risk of the algorithm getting stuck in a local optimum and never being able to reach the required mimimal fitness score.

The problem with waiting for a fixed number of iterations is that there
is no guarantee that the solution achieved after \(n\) iterations doesn't
break any of the hard constraints. To remedy this, Al-Jarrah et al.
took an innovative approach of \emph{restarting and reinitialising} the algorithm
\emph{if} after some fixed number of iterations a satisfactory candidate is not found.
\cite[p.~28]{aljarrah2017} 

\subsection*{Initial Population}

The initial population tends to be randomly generated
\cite[p.~97]{mesghouni2004} \cite[p.~29]{aljarrah2017}.  
A good size of population that researchers are using appears to be
over 100 candidates \cite[p.~101]{mesghouni2004} \cite[p.~16]{timilsina2015}
\cite{hosseinabadi2018}.

Yu Enzhe and Sung Kiseok, for instance, run their GA to solve
a timetabling problem using population sizes of \(25\), \(50\) and
\(100\) and received the best results with \(100\). \cite[p.~715]{yu2002}

Larger populations increase the probability of finding a solution.
\cite[p.~12]{timilsina2015}. The more individuals we have scouting the
fitness landscape, the more likely we are to encounter a peak.
However, it's also important to bear in mind that GAs are typically used 
in very complex problems and each candidate might encode for hundreds
of variables. Assuming that a candidate codes for \(300\) variables and
every variable requires \(32\) bits of memory (enough to represent values
from \(0\) to \(4294967296\)) we end up with each candidate having a
size of \(300 * 32\ bits = 9600\ bits = 1200\ Bytes = 1.2\ KB\).  So it
might not be possible to raise the population size to an arbitrary size.

\section*{Conclusion}

This essay demonstrated how EA can be applied to scheduling.  It
discussed the practices of the research community including: encoding,
initial population, genetic \& selection operators and parameter values.

\begin{thebibliography}{10}

  % BOOKS

  \bibitem{norvig2010}
  Stuart Russel and Peter Norvig,
  \textit{Artificial Intelligence},
  \textit{A Modern Approach},
  Pages 126-129, 
  3rd Ed, 
  Pearson Education
  2010.

  \bibitem{floreano2008}
  Dario Floreano and Claudio Mattiussi,
  \textit{Bio-Inspired Artificial Intelligence},
  \textit{Theories, Methods, and Technologies},
  Pages 1-38,
  MIT Press,
  2008.

  \bibitem{eberhart2007}
  Russel C Eberhart,
  \textit{Computational Intelligence},
  \textit{Concepts to Implementations},
  Pages 103-118,
  Pages 51-68,
  Denise E M Penrose,
  2007.

  \bibitem{heaton2014}
  Jeff Heaton,
  \textit{Artificial Intelligence for Humans},
  \textit{Volume 2: Nature Inspired Algorithms},
  Pages 1-100,
  Heaton Research, Inc,
  2014.

  % PAPERS

  \bibitem{aljarrah2017}
  AlJarrah Mohammad, Atallah AlSawalqah Ahmad and Hamdan Sami, 
  \textit{Developing A Course Timetable System for Academic Departments Using Genetic Algorithm},
  Jordanian Journal of Computers and Information Technology,
  Volume 3,
  No 1,
  Pages 25-36,
  2017.

  \bibitem{diveev2017}
  A I Diveev, O V Bobr,
  \textit{Variational Genetic Algorithm for NP-hard Scheduling Problem Solution},
  Procedia Computer Science,
  Volume 103,
  Pages 52-58,
  2017.

  \bibitem{mesghouni2004}
  Mesghouni Khaled and Hammadi Slim,
  \textit{Evolutionary algorithms for job-shop scheduling}, 
  International Journal of Applied Mathematics and Computer Science,
  Volume 14,
  Pages 91-103,
  2004.

  \bibitem{yu2002}
  Yu Enzhe and Sung Kiseok, 
  \textit{A genetic algorithm for a university weekly courses timetabling problem}, 
  International Transactions in Operational Research,
  Volume 9,
  Pages 703-717,
  2002.

  \bibitem{qaurooni2013}
  Danial Qaurooni, Mohammad-R Akbarzadeh-T,
  \textit{Course timetabling using evolutionary operators},
  Applied Soft Computing,
  Volume 13, 
  Issue 5,
  Pages 2504-2514,
  2013.

  \bibitem{timilsina2015}
  Timilsina Sandesh, Negi Rohit, Khurana Yashika and Seth Jyotsna,
  \textit{Genetically Evolved Solution to Timetable Scheduling Problem},
  International Journal of Computer Applications,
  Volume 114,
  Pages 12-17,
  2015.

  %% \bibitem{pezzella2008}
  %% F Pezzella, G Morganti and G Ciaschetti,
  %% \textit{A genetic algorithm for the Flexible Job-shop Scheduling Problem},
  %% Computers \& Operations Research,
  %% Volume 35, 
  %% Issue 10,
  %% Pages 3202-3212,
  %% 2008.

  \bibitem{may2015}
  Gökan May, Bojan Stahl, Marco Taisch and Vittal Prabhu,
  \textit{Multi-objective genetic algorithm for energy-efficient job shop scheduling},
  International Journal of Production Research
  Volume 53,
  No 23, 
  Pages  7071 - 7089,
  2015.

  \bibitem{hosseinabadi2018}
  Rahmani Hosseinabadi, Ali Asghar, Vahidi Javad, Saemi Behzad, Kumar Arun and Elhoseny Mohamed,
  \textit{Extended Genetic Algorithm for solving open-shop scheduling problem},
  Soft Computing,
  Pages 1-18, 

  \bibitem{solanki2015}
  Sachin V. Solanki
  \textit{Genetic Algorithm Approach for Implementation of Job Scheduling Problem},
  International Journal on Recent and Innovation Trends in Computing and Communication,
  Volume 3,
  Issue 6,
  Pages 3867 - 3872,
  2015.
  
\end{thebibliography}

\end{document}
