%% options: [no]titlepage, twocolumn, landscape, draft
\documentclass[a4paper, 12pt, draft, titlepage]{article}
\usepackage[utf8]{inputenc}
\usepackage[english]{babel}
\date{} % overwrite default format
% \usepackage[backend=biber, style=alphabetic, sorting=ynt]{biblatex}
% \addbibresource{references.bib}
\author{Norbert Logiewa\\nl253}
\title{Applications of Evolutionary Algorithms in \\Scheduling, Planning and Timetabling}

\begin{document}

\maketitle

\section*{Introduction}

Evolutionary algorithms (EA) are a class of procedures that are used when 
finding a solution to a problem using regular means is not
possible or very difficult. This is often because the search space is too
large to be directly traversed so brute-force methods are not possible.
EA allow to find good-enough solution to problems that don't necessarily
need an \emph{optimal} solution.

The aim of this essay is to examine how evolutionary algorithms can be 
applied to schedule activities given a constrained amount of resources.
Efficient scheduling of tasks if of interests to anyone who wishes
to increase their productivity \emph{without} investing additional resources.

Scheduling in the context of this essay will refer to ordering of tasks 
that aims to maximise the productivity. Importantly, the ordering may be 
done in such a way that some activities happen in parallel if business rules
and resources allow for that.
Productivity is the extent to which a schedule helps some organisation 
to achieve their goals. If, for example, we are scheduling for a web development 
agency, we can measure productivity by the lines of code produced by the
developers.

\section*{Case Study}

Let us consider a case study in which a hypothetical university offers
a 200 modules to 16000 students. Each student must take 8 unique
modules. Each module needs to have 2 one-hour lectures during the
working week. Additionally, each of those lectures or can take place
in one of 50 lecture halls. Moreover, each of those lectures may be
given by a lecturer from the relevant department.\footnote{We can assume that the
avg. number of lectures is 50 for every department.} Finally, each lecture
or may take place a time between 9am and 7pm.  \footnote{For the sake of
simplicity I will assume all modules have to be taken i.e.  a situation
in which an unpopular module would be dropped from the curriculum would
not take place.}

Each student would have a combination of modules that they would like to take.
It is the task of the administration to ensure that none of the students 
must be in more than one lecture at the same time. Furthermore, 
none of the lectures can be asked to lecture something outside of their field
of expertise and none can be asked to give more than one lecture at the same time.
Lastly only one lecture can take place in a lecture hall at the same time.
Such restrictions on allocation of resources will be referred to as \emph{business rules}.

Suppose that the hypothetical university would like to improve their
place in league tables. Suppose further,  that from the data they have
collected the shows that students tend to perform better if there is a
gap in between lectures rather than having multiple lectures in a row.

A \emph{candidate solution} that obeys this criterion would be favoured in the
process of selection.

\section*{Solving the Problem}

Clearly, the problem is complicated because there is a large number of independent
variables. Using the brute-force approach would require us to traverse each point 
in this multi-dimensional space which might take too long to complete.

Before we apply EA to the problem it's necessary to define what a solution to
the problem would be. The aim is to find a set of 400 tuples where each tuple
is $(module, lecturer, time, hall)$. Our "solution set" will need to obey the
business rules mentioned above. 

To apply EA we need to initially randomly generate a set of candidate
timetables. This will become the first generation.  On each iteration (next generation), we select the most promising timetables
from the current gene pool (candidate-set of timetables) to enter the next 
gene pool by measuring how well they obey the business rules. This is done
using a \emph{fitness function} that assigns a number to each of the candidates. 

Selection can be done in a number of ways but a popular way of doing
that would be using \emph{tournament selection}.  In this scheme $k$
candidate timetables are compared to each other to produce the best
timetable (the fittest) to compete in the next round.  The population
size has depleted so winners are used to create new individuals using
variation operators in such a way that new individuals have something
in common with the winners but also differ in some way.

The process is repeated until some condition is met \footnote{E.g. 100 round
have been completed or the algorithm has been running for 10 minutes.} at
which point the best timetable (fittest individual) in our candidate-set (population) is
presented as the solution to the timetabling problem.  The idea is that
the quality of the solution \emph{increases} the longer the algorithm
is being run. \footnote{EAs are \emph{greedy algorithms}.}
 
\section*{Summary}

This essay demonstrated how EA can be used to solve common problems
such as timetabling or scheduling. It illustrated how to apply EA using
a case study of a university, it discussed how to encode the candidate
timetables and stated how carry out the algorithm in scenario.

\printbibliography

\end{document}

